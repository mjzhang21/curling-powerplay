\documentclass[12pt]{article}

%% preamble: Keep it clean; only include those you need
\usepackage{amsmath}
\usepackage[margin = 1in]{geometry}
\usepackage{graphicx}
\usepackage{booktabs}
\usepackage{natbib}
\usepackage{subcaption} 

%% for double spacing
\usepackage{setspace}

% for space filling
\usepackage{lipsum}
% highlighting hyper links
\usepackage[colorlinks=true, citecolor=blue]{hyperref}

% Get line numbers for ease of referencing for the reviewers
\usepackage[]{lineno}
\linenumbers*[1]
% patches to make lineno work better with amsmath
\newcommand*\patchAmsMathEnvironmentForLineno[1]{%
        \expandafter\let\csname old#1\expandafter\endcsname\csname 
        #1\endcsname
        \expandafter\let\csname oldend#1\expandafter\endcsname\csname 
        end#1\endcsname
        \renewenvironment{#1}%
        {\linenomath\csname old#1\endcsname}%
        {\csname oldend#1\endcsname\endlinenomath}}%
\newcommand*\patchBothAmsMathEnvironmentsForLineno[1]{%
        \patchAmsMathEnvironmentForLineno{#1}%
        \patchAmsMathEnvironmentForLineno{#1*}}%
\AtBeginDocument{%
        \patchBothAmsMathEnvironmentsForLineno{equation}%
        \patchBothAmsMathEnvironmentsForLineno{align}%
        \patchBothAmsMathEnvironmentsForLineno{flalign}%
        \patchBothAmsMathEnvironmentsForLineno{alignat}%
        \patchBothAmsMathEnvironmentsForLineno{gather}%
        \patchBothAmsMathEnvironmentsForLineno{multline}%
}



%% meta data

\title{Strategic Adaptation to the Power Play in Mixed Doubles Curling}

\author{
Mark Zhang and Jun Yan \\
Department of Statistics, University of Connecticut
}

\begin{document}
\maketitle

\begin{abstract}
        The Power Play in mixed doubles curling was introduced to open
        the sheet and encourage offensive scoring, yet its strategic
        effects remain largely undocumented. Using stone-level
        coordinate and shot assessment data from three World Mixed
        Doubles Curling Championships and one Winter Olympics, this
        study examines strategic behavior associated with Power Play
        usage and response. Analysis of the non-hammer team’s first shot
        against the Power Play identifies three dominant counterplay
        patterns based on spatial stone configurations. These patterns
        differ in execution quality and cumulative score dynamics but
        do not yield statistically significant differences in end-level
        scoring outcomes.
\end{abstract}

\doublespacing

\section{Introduction}\label{sec:intro}

Curling, nicknamed `chess on ice', has rapidly become one of the most popular
Winter Olympic sports \citep{agduman2023review}. With increased international
attention, the World Curling Federation introduced the Free Guard Zone to
matches more competitive and exciting to watch \citep{agduman2023review}. New
rules heavily impact tactical and strategic decisions, as
\citet{agduman2023impact} concludes the Free Guard Zone effectively reduces
blanks and increases scoring. Mixed doubles curling presents a unique strategic
environment with recent rule changes—such as the modified free guard zone,
pre-placed stones, and the Power Play—that alter tactics and increase the
importance of adaptive planning \citep{worldcurlingrules2025}. Developed at the
International Olympic Committee's request for a faster, more variable format,
it lacks extensive data-driven strategies. The Power Play, a key innovation,
can be used once per match when a team has the hammer, and it opens the sheet’s
center by repositioning the pre-placed stones. This shift changes incentives
and necessitates clear guidelines for calling, executing, and responding to the
Power Play.

Strategic and tactical factors in curling is a very niche area, with
\citet{zacharias2024examination} identifying only eleven studies on the topic.
Technology and digital scorebooks have had an important role in collecting
detailed match data, allowing the Japan national team to notice correlation in
shot accuracy and game scores \citep{masui2016informatics}. One popular
question analysts and players have been curious about is whether to blank or
take a single point as the hammer team \citep{kostuk2006paradox}. Previous
studies have used decision analysis and Markov chains to model win
probabilities in each scenario
\citep{willoughby2005analysis,clement2012analysis}. More recent studies have
improved upon Markov modelling, combined it with simulations, and expanded the
use to predicting conceded ends \citep{brenzel2019analysis, fry2024elementary}
Also, multiple linear regression and binary logistic regression have been used
to confirm the clear advantage of possessing the hammer
\citep{park2013analysis,erhan2023exploring}. However, none of these studies
address the dynamics introduced by recent rule changes in mixed doubles
curling, particularly the Power Play. Also, none of the studies utilize
shot-level data such as execution score and type of throw. Because mixed
doubles and the Power Play change strategic incentives, research must extend to
this format to develop data-driven guidelines.

In this study, we use international mixed doubles shot-level data to examine
Power Play ends and quantify how execution quality and tactical choices in the
first three hammer shots relate to hammer-end success, defined as scoring two
or more points in the end. By using the granular shot level type of shot and
execution score, we determine the importance and impact of shot-making on
end-level outcomes. Our goal is to provide evidence that can inform practical
decision-making around calling the Power Play and selecting early-end shot
types.

\section{Data}\label{sec:data}

We used the shot-level dataset compiled and processed by
\citet{ritchie2025opening}. The data were scraped from Curlit Results Booklets
and include World Mixed Doubles Curling Championship matches from 2016--2025 as
well as the Olympic Winter Games in 2018 and 2022 \citep{ritchie2025opening}.
The dataset contains 64{,}471 recorded shots and 167 match-, end-, and
stone-level variables. Key variables include match and end identifiers, end
number, hammer designation, team identities, a Power Play indicator,
pre-/during-/post-end scoring states, and shot-level measures such as shot
number, coordinates, declared shot type, and a five-level execution score (0,
25, 50, 75, 100), where higher values indicate better execution.

\section{Methods}\label{sec:meth}

We focused on the first three hammer shots in Power Play ends. To ensure
comparability across ends, we restricted the sample to ends with a complete
shot structure and excluded conceded ends. We then filtered the dataset to
Power Play ends and to hammer team shots 2, 4, and 6. Hammer and non-hammer
team variables were created by matching the hammer indicator to team identities
so that score states and covariates were consistently defined from the hammer
team's perspective.

To improve interpretability and avoid sparse categories, we grouped declared
shot types into four throw groups. Draws and freezes were classified as Draw;
guards and fronts as Guard; wick-soft-peeling and raise as Peel/Raise; and
take-out, double take-out, hit-and-roll, promotion take-out, and clearing as
Removal. Because Power Play ends occur disproportionately in later ends, we
binned end number into five groups: $\le4$, 5, 6, 7, and 8. We also constructed
contextual covariates. Score difference before the end was defined as the
hammer team's score minus the non-hammer team's score at the start of the end.
Team strength was calculated using a weighted average of hammer, force, and
steal efficiencies computed across all ends played by each nation
\citep{ritchie2025opening}, and strength difference was defined as hammer-team
strength minus non-hammer-team strength. For modeling, execution scores were
scaled by 25 (so one unit corresponds to a 25-point increase), score difference
was mean-centered, and strength difference was standardized.

The primary outcome was hammer-end success, defined as the hammer team scoring
two or more points in the end. To estimate shot-specific associations,
execution score and throw group were pivoted to create separate covariates for
shots 2, 4, and 6 (e.g., shot-specific execution and throw-group indicators).
We then fit a multivariable logistic regression model with hammer-end success
as the binary outcome and shot-specific execution, shot-specific throw group,
end group, pre-end score difference, and strength difference as predictors. For
each shot, Draw was used as the reference throw group, and end 6 was used as
the reference end group. Results are reported as odds ratios with 95\%
confidence intervals.

\section{Results}\label{sec:resu}

Across Power Play ends, hammer-team shot-call distributions differed by shot
number (Table~\ref{tab:pp_throw_dist}). On hammer shot 2, calls were primarily
Draw (n = 976) and Guard (n = 140), with few Removal attempts (n = 6) and
Peel/Raise calls (n = 163). On hammer shot 4, calls included Removal (n = 573),
Draw (n = 499), Guard (n = 34), and Peel/Raise (n = 182). On hammer shot 6,
calls included Removal (n = 679), Draw (n = 453), Guard (n = 41), and
Peel/Raise (n = 116).

\begin{table}
        \centering
        \caption{Distribution of hammer-team shot calls by shot number in Power Play ends. Entries are count (within-shot \%).}
        \label{tab:pp_throw_dist}
        \begin{tabular}{lccccc}
                \toprule
                       & Draw         & Guard        & Removal      & Peel/Raise   & Total \\
                \midrule
                Shot 2 & 976 (76.0\%) & 140 (10.9\%) & 6 (0.5\%)    & 163 (12.7\%) & 1285  \\
                Shot 4 & 499 (38.7\%) & 34 (2.6\%)   & 573 (44.5\%) & 182 (14.1\%) & 1288  \\
                Shot 6 & 453 (35.1\%) & 41 (3.2\%)   & 679 (52.7\%) & 116 (9.0\%)  & 1289  \\
                Total  & 1928         & 215          & 1258         & 461          & 3862  \\
                \bottomrule
        \end{tabular}
\end{table}

A multivariable logistic regression model was fit for hammer-end success
(scoring $\ge 2$) using shot-specific execution and shot call for hammer shots
2, 4, and 6, while adjusting for end group, pre-end score difference, and
relative team strength (Table~\ref{tab:pp_hammer_success}). Execution variables
were scaled such that a one-unit increase corresponds to a 25-point increase in
execution score.

\begin{table}[t]
        \centering
        \caption{Multivariable logistic regression for Power Play hammer-end success (scoring $\ge$ 2). Execution is per 25 points.}
        \label{tab:pp_hammer_success}
        \begin{tabular}{lcc}
                \toprule
                                                  & OR (95\% CI)      & $p$      \\
                \midrule
                Shot 2 call: Guard (vs Draw)      & 0.55 [0.36, 0.83] & 0.004    \\
                Shot 2 call: Peel/Raise (vs Draw) & 0.76 [0.51, 1.13] & 0.179    \\
                Shot 4 call: Guard (vs Draw)      & 1.31 [0.60, 2.85] & 0.499    \\
                Shot 4 call: Removal (vs Draw)    & 0.92 [0.69, 1.22] & 0.556    \\
                Shot 4 call: Peel/Raise (vs Draw) & 0.71 [0.48, 1.03] & 0.073    \\
                Shot 6 call: Guard (vs Draw)      & 2.67 [1.29, 5.52] & 0.008    \\
                Shot 6 call: Removal (vs Draw)    & 0.88 [0.67, 1.16] & 0.372    \\
                Shot 6 call: Peel/Raise (vs Draw) & 1.05 [0.67, 1.63] & 0.845    \\
                End group $\le$4 (vs 6)           & 0.42 [0.25, 0.70] & $<0.001$ \\
                End group 5 (vs 6)                & 0.93 [0.63, 1.38] & 0.733    \\
                End group 7 (vs 6)                & 1.16 [0.88, 1.54] & 0.298    \\
                End group 8 (vs 6)                & 0.39 [0.22, 0.67] & $<0.001$ \\
                Shot 6 execution (per +25)        & 1.42 [1.29, 1.57] & $<0.001$ \\
                Pre-end score diff (per +1)       & 0.95 [0.91, 1.01] & 0.085    \\
                Strength diff (per +1 SD)         & 1.43 [1.24, 1.65] & $<0.001$ \\
                \bottomrule
        \end{tabular}
\end{table}

Execution was positively associated with hammer-end success for all three
shots. The estimated association was largest for shot 6 execution (OR $=1.42$,
95\% CI $[1.29,\,1.57]$; Table~\ref{tab:pp_hammer_success}). Shot-call
associations were smaller and less precisely estimated. Relative to a Draw call
on shot 2, a Guard call was associated with lower odds of success (OR $=0.55$,
95\% CI $[0.36,\,0.83]$), while Peel/Raise on shot 2 did not differ from Draw
(OR $=0.76$, 95\% CI $[0.51,\,1.13]$). For shot 6, calling a Guard rather than
a Draw was associated with higher odds of success (OR $=2.67$, 95\% CI
$[1.29,\,5.52]$), whereas Removal and Peel/Raise on shot 6 did not differ
meaningfully from Draw. For shot 4, Guard and Removal did not differ from Draw,
while Peel/Raise showed lower odds of success with borderline evidence (OR
$=0.71$, 95\% CI $[0.48,\,1.03]$).

Match-context covariates showed consistent patterns. Relative to end group 6,
success odds were lower in early ends ($\le 4$; OR $=0.42$, 95\% CI
$[0.25,\,0.70]$) and in end 8 (OR $=0.39$, 95\% CI $[0.22,\,0.67]$). Greater
relative team strength was associated with higher odds of success (OR $=1.43$
per one standard deviation increase in strength difference, 95\% CI
$[1.24,\,1.65]$). Pre-end score difference was negatively associated with
success (OR $=0.95$ per one-point increase, 95\% CI $[0.91,\,1.01]$). Full
model estimates are reported in Table~\ref{tab:pp_hammer_success}.

\section{Discussion}\label{sec:disc}

\bibliographystyle{apalike}
\clearpage
\bibliography{refs.bib}

\end{document}