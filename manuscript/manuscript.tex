\documentclass[12pt]{article}

%% preamble: Keep it clean; only include those you need
\usepackage{amsmath}
\usepackage[margin = 1in]{geometry}
\usepackage{graphicx}
\usepackage{booktabs}
\usepackage{natbib}
\usepackage{subcaption} 

%% for double spacing
\usepackage{setspace}

% for space filling
\usepackage{lipsum}
% highlighting hyper links
\usepackage[colorlinks=true, citecolor=blue]{hyperref}

% Get line numbers for ease of referencing for the reviewers
\usepackage[]{lineno}
\linenumbers*[1]
% patches to make lineno work better with amsmath
\newcommand*\patchAmsMathEnvironmentForLineno[1]{%
        \expandafter\let\csname old#1\expandafter\endcsname\csname 
        #1\endcsname
        \expandafter\let\csname oldend#1\expandafter\endcsname\csname 
        end#1\endcsname
        \renewenvironment{#1}%
        {\linenomath\csname old#1\endcsname}%
        {\csname oldend#1\endcsname\endlinenomath}}%
\newcommand*\patchBothAmsMathEnvironmentsForLineno[1]{%
        \patchAmsMathEnvironmentForLineno{#1}%
        \patchAmsMathEnvironmentForLineno{#1*}}%
\AtBeginDocument{%
        \patchBothAmsMathEnvironmentsForLineno{equation}%
        \patchBothAmsMathEnvironmentsForLineno{align}%
        \patchBothAmsMathEnvironmentsForLineno{flalign}%
        \patchBothAmsMathEnvironmentsForLineno{alignat}%
        \patchBothAmsMathEnvironmentsForLineno{gather}%
        \patchBothAmsMathEnvironmentsForLineno{multline}%
}



%% meta data

\title{Strategic Adaptation to the Power Play in Mixed Doubles Curling}
\author{Jun Yan\\
  Department of Statistics\\
  University of Connecticut
}

\begin{document}
\maketitle

\begin{abstract}
  The power play in mixed doubles curling was introduced to open
  the sheet and encourage offensive scoring, yet its strategic
  effects remain largely undocumented. Using stone-level
  coordinate and shot assessment data from three World Mixed
  Doubles Curling Championships and one Winter Olympics, this
  study examines strategic behavior associated with power play
  usage and response. Analysis of the non-hammer team’s first shot
  against the power play identifies three dominant counterplay
  patterns based on spatial stone configurations. These patterns
  differ in execution quality and cumulative score dynamics but
  do not yield statistically significant differences in end-level
  scoring outcomes.
\end{abstract}

\doublespacing

\section{Introduction}
\label{sec:intro}
Mixed doubles curling presents a unique strategic environment with recent rule
changes—such as the modified free guard zone, pre-placed stones, and the power
play—that alter tactics and increase the importance of adaptive planning
\citep{WorldCurlingRules2025}. Developed at the International Olympic
Committee's request for a faster, more variable format, it lacks extensive
data-driven strategies. The power play, a key innovation, can be used once per
match when a team has the hammer, and it opens the sheet’s center by
repositioning the pre-placed stones. This shift changes incentives and
necessitates clear guidelines for calling, executing, and responding to the
power play.

Strategy, curling physics, and robotics are major research areas in curling
\citep{zacharias2024examination, maeno2016assignments, gwon2020path}. Research
on curling strategy uses decision analysis, logistic regression, and deep
reinforcement learning \citep{willoughby2005analysis, clement2012analysis,
  han2022game}. For example, \citet{willoughby2005analysis} and
\citet{clement2012analysis} address the long-debated question of whether
blanking or scoring one is the better choice in the seventh end. Furthermore,
\citet{park2013curling} confirms the advantage of possessing the hammer with
statistical evidence. However, these studies do not address the dynamics
introduced by recent rule changes in mixed doubles curling, particularly the
power play. Because mixed doubles and the power play change strategic
incentives, research must extend to this format to develop data-driven
guidelines.

Using data from three World Championships and one Winter Olympics, this paper
offers the first analysis of strategic behavior surrounding the power play in
professional mixed doubles curling. We show that late-phase behavior shifts
toward defense in power play ends, with the largest and most persistent shift
coming from the non-calling team. We quantify timing effects, provide
team-level descriptive strategy profiles, and validate the behavioral proxy
with stone-removal outcomes. We then focus on non-hammer team behavior by
analyzing common counter power play shot types and show that different openings
do not significantly alter end-level scoring outcomes, even though they produce
meaningful differences in execution quality and cumulative score dynamics.
Despite teams systematically varying shot choice with game context, the data
provide no evidence that any single counter-strategy is objectively dominant
when opening an end against the power play.

\section{Data}
\label{sec:data}
We use Curlit's mixed doubles tracking data, which records shot-by-shot stone
positions and analyst annotations across three World Mixed Doubles
Championships and one Winter Olympics. The dataset contains 26,730 shot rows
and includes unique identifiers for competition, game, end, team, player, and
shot order, so that each match can be reconstructed chronologically. In
addition to x--y stone coordinates after every throw, Curlit provides
analyst-graded fields for intended task and execution quality, which we use to
classify behavior. End results are recorded for both teams, enabling
reconstruction of score progression and pre-end score differential.

Before analysis, we combined datasets, removed irrelevant rows, standardized
coordinates, and kept only the first shot of each right-formation power play
end. We swapped the pre-placed stone columns so the first coordinates represent
the non-hammer guard and the second represent the hammer team stone near the
button. Rows with missing data were dropped, which removes shots that eliminate
stones before the Modified Free Guard Zone.

\section{Methods}
\label{sec:meth}
To explore strategic patterns in first-shot power plays, we applied PCA, t-SNE,
and UMAP, then clustered with HDBSCAN. This reduces six-dimensional coordinates
(x and y for pre-placed and first stones) to two dimensions while revealing
common tactical placements. Cluster quality was assessed using silhouette
scores, membership probabilities, and coverage metrics. Curling sheet scatter
plots illustrate typical stone positions for each cluster and provide a view of
different board configurations.

Performance differences across clusters were evaluated with Welch’s ANOVA and
bootstrap confidence intervals to handle skew and unequal variances; Games--
Howell tests identified which clusters differed. To examine international
variation, first non-hammer shots in left and right formation power plays were
summarized with histograms and bar charts. Only teams with more than twenty
power play ends were included. Similar shot types were grouped, and the top
five countries for each type were plotted to reveal national tendencies in
responding to power plays.

\section{Results}
\label{sec:resu}
The HDBSCAN clustering produced four distinct clusters, capturing board
patterns in first-shot power play ends. The coverage rate, DBCV score, and
silhouette score (0.604) indicate a reasonably strong clustering structure
despite some noise. Clusters 0–3 had mean membership probabilities of 0.773,
0.753, 0.891, and 0.982, respectively, indicating that Cluster 3 contained the
most tightly grouped points, while Clusters 0 and 1 had weaker assignments.
Visualizations on PCA, t-SNE, and UMAP confirmed well-separated clusters with
minimal overlap (Figure~\ref{fig:fs_tSNE}). Overall, the clustering identifies
meaningful groups that can be analyzed for strategic differences.

\begin{figure}[t]
  \centering
  \includegraphics[width=1.0\textwidth]{fs_t-SNE.pdf}
  \caption{“t-SNE visualization of the first-shot power-play left-formation
    dataset in two dimensions.
    Each point represents a sample, colored by cluster.
    Clusters indicate similarity in the feature space.”}
  \label{fig:fs_tSNE}
\end{figure}

To further characterize these clusters, we examined the spatial distribution of
stone placements and shot types. Plotting the stone coordinates on a curling
sheet shows three distinct board configurations (Figure~\ref{fig:C_all}). Some
stones interact with pre-placed guards, while others leave them untouched and
prioritize placement near the house or in the front. Cluster 3 represented a
central draw pattern but was excluded from further analysis due to small sample
size ($n$ = 7). We next examined the distribution of shot types within each
cluster. Cluster 0 exhibited a roughly equal proportion of raise/tapbacks and
wick/soft-peeling shots, Cluster 1 consisted entirely of draws, and Cluster 2
had an equal mix of front and guard placements. Together, these spatial and
shot-type patterns highlight the primary variations in early power play
strategies across clusters.

\begin{figure}[t]
  \centering

  \begin{subfigure}[b]{0.32\textwidth}
    \centering
    \includegraphics[width=\textwidth]{C0.pdf}
    \caption{Cluster 0: Tick shot}
    \label{fig:C0}
  \end{subfigure}
  \hfill
  \begin{subfigure}[b]{0.32\textwidth}
    \centering
    \includegraphics[width=\textwidth]{C1.pdf}
    \caption{Cluster 1: Draw}
    \label{fig:C1}
  \end{subfigure}
  \hfill
  \begin{subfigure}[b]{0.32\textwidth}
    \centering
    \includegraphics[width=\textwidth]{C2.pdf}
    \caption{Cluster 2: Front/Guard}
    \label{fig:C2}
  \end{subfigure}

  \caption{Coordinate visualizations after the first shot in the power-play
    left formation. Each panel highlights a different cluster, which can be
    identified as shot types: tick (Panel A), draw around the guard (Panel B),
    and front/guard (Panel C).}
  \label{fig:C_all}
\end{figure}

Finally, we assessed whether cluster membership was associated with differences
in game outcomes. Cumulative score difference differed significantly across
clusters (Welch’s ANOVA, $p=0.017$; Table~\ref{tab:cluster_test}), with the
highest mean observed in C1 and the lowest in C0. Post-hoc Games--Howell tests
indicated that this effect was driven by a significant difference between C0
and C1, whereas C2 did not differ significantly from either cluster. Points
also varied across clusters (Welch’s ANOVA, $p=0.005$), with a significant
difference observed between C1 and C2 only. In contrast, hammer and non-hammer
scores did not differ significantly across clusters. Overall, these results
indicate that cluster structure was associated with variation in overall
pre-end score differences and execution score, but not with hammer-specific
performance.

\begin{table}[t]
  \centering
  \caption{Group means (with 95\% bootstrap confidence intervals) for each
    outcome variable by HDBSCAN cluster. p-values are from Welch’s one-way
    ANOVA. Significant pairwise differences were assessed using Games–Howell
    post hoc tests.}
  \label{tab:cluster_test}
  \small
  \setlength{\tabcolsep}{4pt}
  \begin{tabular}{lcccc}
    \hline
    Variable & C0 ($n$=27)       & C1 ($n$=76) & C2 ($n$=65) & $p$-value \\
    \hline
    Cumulative score difference
             & 3.44 (1.93--5.00)
             & 6.12 (5.21--7.03)
             & 4.98 (3.86--6.09)
             & 0.017                                                     \\
    Points
             & 3.63 (3.26--3.93)
             & 3.34 (3.17--3.50)
             & 3.74 (3.57--3.88)
             & 0.005                                                     \\
    Hammer score
             & 1.30 (0.85--1.78)
             & 1.44 (1.16--1.73)
             & 1.52 (1.18--1.87)
             & 0.769                                                     \\
    Non-hammer score
             & 0.37 (0.15--0.63)
             & 0.37 (0.21--0.56)
             & 0.31 (0.18--0.47)
             & 0.833                                                     \\
    \hline
  \end{tabular}

  \begin{flushleft}
    \footnotesize
    Notes: Values are means with 95\% confidence intervals in parentheses.
    Pairwise Games--Howell comparisons for significant variables:
    Cumulative score difference—C0 vs C1 ($p=0.018$), C0 vs C2 ($p=0.281$),
    C1 vs C2 ($p=0.279$);
    Points—C0 vs C1 ($p=0.326$), C0 vs C2 ($p=0.845$), C1 vs C2 ($p=0.004$).
  \end{flushleft}
\end{table}

\section{Discussion}
\label{sec:disc}
Analysts have observed different responses for countering the power play
\citep{rocks2021mixed}, but no studies identify which first shots teams play
and how they affect performance and context metrics. By clustering board
positions after the non-hammer team’s first shot, three patterns emerge in
mixed doubles championships: tick, guard, and draw around the guard. These
patterns matter because the first shot shapes subsequent shot selection, and
identifying international responses helps teams prepare for specific
opponents.

Welch ANOVA shows differences in cumulative score difference and points in
non-missed shots, while hammer and non-hammer scores do not differ. Games--
Howell tests indicate that cumulative score differences are driven by contrasts
between the tick and draw-around-the-guard clusters. All shots are utilized
when the hammer team has the score advantage, but tick shots are more common
when the lead is small, whereas draws are favored when the hammer team leads by
a margin. These patterns suggest teams take fewer risks when trailing modestly.
Since hammer and non-hammer scores do not vary across clusters, shot choice may
not strongly influence end-level performance. This suggests non-hammer teams
select the starting player based on execution skill. Limitations include small
power play sample size and skewed features, particularly in non-hammer score.

When considering all power play ends, countries differ in their first-shot
strategies. The bar chart of task-group proportions by nation highlights these
variations (Figure~\ref{fig:noc_pptask}). While draws and tick shots are common
across teams, some national tendencies stand out, reflecting different
approaches to countering the power play. These patterns align with analysts’
expectations about strategic variation and demonstrate how teams adapt their
play style in response to the power play. Further analysis of subsequent shots
and early shot sequences could provide additional insight into how teams build
on initial positioning throughout the end.

\begin{figure}[t]
  \centering
  \includegraphics[width=\textwidth]{noc_pptask.pdf}
  \caption{“Proportions of grouped tasks hit by country. Data includes all
    power play ends and is limited to the first stone. Only the top five teams
    with more than 20 power play ends are shown.”}
  \label{fig:noc_pptask}
\end{figure}

At the shot level, clustering of the non-hammer team’s first response reveals
three dominant counterplay patterns: tick, draw around the guard, and center
guard. We conclude that while execution quality and score difference vary, the
first stone shot selection does not impact non-hammer or hammer team end
scores. Teams show clear first-stone preferences, but other factors likely
drive power play end-level success.

\bibliographystyle{apalike}
\clearpage
\bibliography{refs.bib}

\end{document}