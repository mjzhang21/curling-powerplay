\documentclass[12pt]{article}

%% preamble: Keep it clean; only include those you need
\usepackage{amsmath}
\usepackage[margin = 1in]{geometry}
\usepackage{graphicx}
\usepackage{booktabs}
\usepackage{natbib}
\usepackage{subcaption} 

%% for double spacing
\usepackage{setspace}

% for space filling
\usepackage{lipsum}
% highlighting hyper links
\usepackage[colorlinks=true, citecolor=blue]{hyperref}

% Get line numbers for ease of referencing for the reviewers
\usepackage[]{lineno}
\linenumbers*[1]
% patches to make lineno work better with amsmath
\newcommand*\patchAmsMathEnvironmentForLineno[1]{%
        \expandafter\let\csname old#1\expandafter\endcsname\csname 
        #1\endcsname
        \expandafter\let\csname oldend#1\expandafter\endcsname\csname 
        end#1\endcsname
        \renewenvironment{#1}%
        {\linenomath\csname old#1\endcsname}%
        {\csname oldend#1\endcsname\endlinenomath}}%
\newcommand*\patchBothAmsMathEnvironmentsForLineno[1]{%
        \patchAmsMathEnvironmentForLineno{#1}%
        \patchAmsMathEnvironmentForLineno{#1*}}%
\AtBeginDocument{%
        \patchBothAmsMathEnvironmentsForLineno{equation}%
        \patchBothAmsMathEnvironmentsForLineno{align}%
        \patchBothAmsMathEnvironmentsForLineno{flalign}%
        \patchBothAmsMathEnvironmentsForLineno{alignat}%
        \patchBothAmsMathEnvironmentsForLineno{gather}%
        \patchBothAmsMathEnvironmentsForLineno{multline}%
}



%% meta data

\title{Strategic Adaptation to the Power Play in Mixed Doubles Curling}

\author{
Mark Zhang and Jun Yan \\
Department of Statistics, University of Connecticut
}

\begin{document}
\maketitle

\begin{abstract}
        The Power Play in mixed doubles curling was introduced to open the sheet and
        encourage offensive scoring, yet evidence-based guidance for calling and
        executing Power Play ends remains limited. Using stone-level coordinate and
        shot assessment data from World Mixed Doubles Curling Championships (2016--2025)
        and the Olympic Winter Games (2018, 2022), we examine how early hammer-team shot
        execution and shot-call choices relate to Power Play end outcomes. We focus on
        hammer shots 2, 4, and 6, group declared shot types into four throw groups
        (Draw, Guard, Removal, and Peel/Raise), and define hammer-end success as scoring
        two or more points. In a multivariable logistic regression adjusting for end
        timing, pre-end score difference, and relative team strength, higher execution
        on shots 2, 4, and 6 is associated with higher odds of success, with the largest
        association for shot 6 execution. Shot-call associations are smaller and depend
        on shot number: relative to Draw, a Guard call on shot 2 and a Peel/Raise call
        on shot 4 are associated with lower odds of scoring $\ge 2$. Overall, Power Play
        conversion appears primarily execution-driven, with call differences most
        relevant early and mid-end when teams are establishing or reshaping scoring
        structure.
\end{abstract}

\doublespacing

\section{Introduction}\label{sec:intro}

Curling, nicknamed `chess on ice', has rapidly become one of the most popular
Winter Olympic sports \citep{agduman2023review}. With increased international
attention, the World Curling Federation introduced the Free Guard Zone to
matches more competitive and exciting to watch \citep{agduman2023review}. New
rules heavily impact tactical and strategic decisions, as
\citet{agduman2023impact} concludes the Free Guard Zone effectively reduces
blanks and increases scoring. Mixed doubles curling presents a unique strategic
environment with recent rule changes—such as the modified free guard zone,
pre-placed stones, and the Power Play—that alter tactics and increase the
importance of adaptive planning \citep{worldcurlingrules2025}. Developed at the
International Olympic Committee's request for a faster, more variable format,
it lacks extensive data-driven strategies. The Power Play, a key innovation,
can be used once per match when a team has the hammer, and it opens the sheet’s
center by repositioning the pre-placed stones. This shift changes incentives
and necessitates clear guidelines for calling, executing, and responding to the
Power Play.

Strategic and tactical factors in curling is a very niche area, with
\citet{zacharias2024examination} identifying only eleven studies on the topic.
Technology and digital scorebooks have had an important role in collecting
detailed match data, allowing the Japan national team to notice correlation in
shot accuracy and game scores \citep{masui2016informatics}. One popular
question analysts and players have been curious about is whether to blank or
take a single point as the hammer team \citep{kostuk2006paradox}. Previous
studies have used decision analysis and Markov chains to model win
probabilities in each scenario
\citep{willoughby2005analysis,clement2012analysis}. More recent studies have
improved upon Markov modelling, combined it with simulations, and expanded the
use to predicting conceded ends \citep{brenzel2019analysis, fry2024elementary}
Also, multiple linear regression and binary logistic regression have been used
to confirm the clear advantage of possessing the hammer
\citep{park2013analysis,erhan2023exploring}. However, none of these studies
address the dynamics introduced by recent rule changes in mixed doubles
curling, particularly the Power Play. Also, none of the studies utilize
shot-level data such as execution score and type of throw. Because mixed
doubles and the Power Play change strategic incentives, research must extend to
this format to develop data-driven guidelines.

In this study, we use international mixed doubles shot-level data to examine
Power Play ends and quantify how execution quality and tactical choices in the
first three hammer shots relate to hammer-end success, defined as scoring two
or more points in the end. By using the granular shot level type of shot and
execution score, we determine the importance and impact of shot-making on
end-level outcomes. Our goal is to provide evidence that can inform practical
decision-making around calling the Power Play and selecting early-end shot
types.

\section{Data}\label{sec:data}

We used the shot-level dataset compiled and processed by
\citet{ritchie2025opening}. The data were scraped from Curlit Results Booklets
and include World Mixed Doubles Curling Championship matches from 2016--2025 as
well as the Olympic Winter Games in 2018 and 2022 \citep{ritchie2025opening}.
The dataset contains 64{,}471 recorded shots and 167 match-, end-, and
stone-level variables. Key variables include match and end identifiers, end
number, hammer designation, team identities, a Power Play indicator,
pre-/during-/post-end scoring states, and shot-level measures such as shot
number, coordinates, declared shot type, and a five-level execution score (0,
25, 50, 75, 100), where higher values indicate better execution.

\section{Methods}\label{sec:meth}

We focused on the first three hammer shots in Power Play ends. To ensure
comparability across ends, we restricted the sample to ends with a complete
shot structure and excluded conceded ends. We then filtered the dataset to
Power Play ends and to hammer team shots 2, 4, and 6. Hammer and non-hammer
team variables were created by matching the hammer indicator to team identities
so that score states and covariates were consistently defined from the hammer
team's perspective.

To improve interpretability and avoid sparse categories, we grouped declared
shot types into four throw groups. Draws and freezes were classified as Draw;
guards and fronts as Guard; wick-soft-peeling and raise as Peel/Raise; and
take-out, double take-out, hit-and-roll, promotion take-out, and clearing as
Removal. Because Power Play ends occur disproportionately in later ends, we
binned end number into five groups: $\le4$, 5, 6, 7, and 8. We also constructed
contextual covariates. Score difference before the end was defined as the
hammer team's score minus the non-hammer team's score at the start of the end.
Team strength was calculated using a weighted average of hammer, force, and
steal efficiencies computed across all ends played by each nation
\citep{ritchie2025opening}, and strength difference was defined as hammer-team
strength minus non-hammer-team strength. For modeling, execution scores were
scaled by 25 (so one unit corresponds to a 25-point increase), score difference
was mean-centered, and strength difference was standardized.

The primary outcome was hammer-end success, defined as the hammer team scoring
two or more points in the end. To estimate shot-specific associations,
execution score and throw group were pivoted to create separate covariates for
shots 2, 4, and 6 (e.g., shot-specific execution and throw-group indicators).
We then fit a multivariable logistic regression model with hammer-end success
as the binary outcome and shot-specific execution, shot-specific throw group,
end group, pre-end score difference, and strength difference as predictors. For
each shot, Draw was used as the reference throw group, and end 6 was used as
the reference end group. Results are reported as odds ratios with 95\%
confidence intervals.

\section{Results}\label{sec:resu}

Across Power Play ends, hammer-team shot-call distributions differed by shot
number (Table~\ref{tab:pp_throw_dist}). On hammer shot 2, calls were primarily
Draw (n = 976) and Guard (n = 140), with few Removal attempts (n = 6) and
Peel/Raise calls (n = 163). On hammer shot 4, calls included Removal (n = 573),
Draw (n = 499), Guard (n = 34), and Peel/Raise (n = 182). On hammer shot 6,
calls included Removal (n = 679), Draw (n = 453), Guard (n = 41), and
Peel/Raise (n = 116).

\begin{table}
        \centering
        \caption{Distribution of hammer-team shot calls by shot number in Power Play ends. Entries are count (within-shot \%).}
        \label{tab:pp_throw_dist}
        \begin{tabular}{lccccc}
                \toprule
                       & Draw         & Guard        & Removal      & Peel/Raise   & Total \\
                \midrule
                Shot 2 & 976 (76.0\%) & 140 (10.9\%) & 6 (0.5\%)    & 163 (12.7\%) & 1285  \\
                Shot 4 & 499 (38.7\%) & 34 (2.6\%)   & 573 (44.5\%) & 182 (14.1\%) & 1288  \\
                Shot 6 & 453 (35.1\%) & 41 (3.2\%)   & 679 (52.7\%) & 116 (9.0\%)  & 1289  \\
                Total  & 1928         & 215          & 1258         & 461          & 3862  \\
                \bottomrule
        \end{tabular}
\end{table}

A multivariable logistic regression model was fit for hammer-end success
(scoring $\ge 2$) using shot-specific execution and shot call for hammer shots
2, 4, and 6, while adjusting for end group, pre-end score difference, and
relative team strength (Table~\ref{tab:pp_hammer_success}). Execution variables
were scaled so that a one-unit increase corresponds to a 25-point increase in
execution score.

\begin{table}
        \centering
        \caption{Multivariable logistic regression for Power Play hammer-end success (scoring $\ge$ 2). Execution is per 25 points.}
        \label{tab:pp_hammer_success}
        \begin{tabular}{lcc}
                \toprule
                                                  & OR (95\% CI)      & p        \\
                \midrule
                Shot 2 call: Guard (vs Draw)      & 0.60 [0.38, 0.95] & 0.030    \\
                Shot 2 call: Peel/Raise (vs Draw) & 0.73 [0.47, 1.14] & 0.165    \\
                Shot 4 call: Guard (vs Draw)      & 1.16 [0.50, 2.70] & 0.725    \\
                Shot 4 call: Removal (vs Draw)    & 0.79 [0.57, 1.09] & 0.158    \\
                Shot 4 call: Peel/Raise (vs Draw) & 0.60 [0.39, 0.92] & 0.018    \\
                Shot 6 call: Guard (vs Draw)      & 1.58 [0.71, 3.52] & 0.259    \\
                Shot 6 call: Removal (vs Draw)    & 0.97 [0.71, 1.31] & 0.822    \\
                Shot 6 call: Peel/Raise (vs Draw) & 1.07 [0.64, 1.77] & 0.803    \\
                End group $\le$4 (vs 6)           & 0.45 [0.25, 0.81] & 0.008    \\
                End group 5 (vs 6)                & 0.93 [0.60, 1.44] & 0.737    \\
                End group 7 (vs 6)                & 1.17 [0.85, 1.60] & 0.344    \\
                End group 8 (vs 6)                & 0.43 [0.23, 0.78] & 0.006    \\
                Shot 2 execution (per +25)        & 1.15 [1.04, 1.27] & 0.006    \\
                Shot 4 execution (per +25)        & 1.14 [1.03, 1.27] & 0.015    \\
                Shot 6 execution (per +25)        & 1.38 [1.24, 1.53] & $<0.001$ \\
                Pre-end score diff (per +1)       & 0.98 [0.92, 1.04] & 0.418    \\
                Strength diff (per +1 SD)         & 1.25 [1.07, 1.46] & 0.005    \\
                \bottomrule
        \end{tabular}
\end{table}

Execution was positively associated with hammer-end success for all three
shots, with the largest association for shot 6 execution (OR $= 1.38$, 95\% CI
$[1.24,\,1.53]$; Table~\ref{tab:pp_hammer_success}). Shot-call associations
were smaller and generally less precisely estimated. Relative to a Draw call on
shot 2, a Guard call was associated with lower odds of success (OR $= 0.60$,
95\% CI $[0.38,\,0.95]$), whereas Peel/Raise on shot 2 did not differ
meaningfully from Draw (OR $= 0.73$, 95\% CI $[0.47,\,1.14]$). For shot 6, call
type was not strongly associated with success: Guard (OR $= 1.58$, 95\% CI
$[0.71,\,3.52]$), Removal (OR $= 0.97$, 95\% CI $[0.71,\,1.31]$), and
Peel/Raise (OR $= 1.07$, 95\% CI $[0.64,\,1.77]$) did not differ meaningfully
from Draw. For shot 4, Peel/Raise was associated with lower odds of success (OR
$= 0.60$, 95\% CI $[0.39,\,0.92]$), while Guard and Removal did not differ from
Draw.

Match-context covariates showed consistent patterns. Relative to end group 6,
success odds were lower in early ends ($\le 4$; OR $= 0.45$, 95\% CI
$[0.25,\,0.81]$) and in end 8 (OR $= 0.43$, 95\% CI $[0.23,\,0.78]$). Greater
relative team strength was associated with higher odds of success (OR $= 1.25$
per one standard deviation increase in strength difference, 95\% CI
$[1.07,\,1.46]$). Pre-end score difference was not meaningfully associated with
success (OR $= 0.98$ per one-point increase, 95\% CI $[0.92,\,1.04]$). Full
model estimates are reported in Table~\ref{tab:pp_hammer_success}.

\section{Discussion}\label{sec:disc}

Our findings highlight several findings about shot type, quality and context in
determining whether the hammer team scores two or more points in a Power Play
end. First, hammer end success is most strongly associated with shot execution,
supporting the view that mixed doubles is a low-margin format which rewards
precise shot-making. Although shot 6 execution has the largest association,
execution on shots 2 and 4 also contributes to hammer-end success. This implies
that the Power Play advantage is realized primarily through consistent shot
quality rather than through a single prescribed early-end throw sequence.

Second, shot-call categories appear to play different strategic roles depending
on shot number. Early in the end (shot 2), call choices are closely tied to
establishing scoring structure under the shifted pre-placed stones. In this
setting, Draw-based development may better support immediate scoring pressure,
whereas early Guard usage may reflect more conservative planning or a response
to an unfavorable opening position. Mid-end (shot 4), Peel/Raise attempts may
represent higher-variance efforts to reshape the scoring layout when simpler
options are less available. Late in the end (shot 6), the call label provides
less incremental information once the end state is largely determined, and the
outcome is more sensitive to whether the chosen plan is executed precisely.

These interpretations are limited by the information available in the model.
Throw types are selected in response to the evolving score differential and
board position. Incorporating spatial coordinates would allow for a more
detailed look into board positioning versus identified throw type. Also
modelling full end score rather than a binary $\ge 2$ outcome may provide more
information on the distribution of outcomes.

Match context also influenced Power Play conversion. Lower success odds in
early ends and in end 8 suggest that timing shapes strategic choices and the
scoring of two or more points. Power Play used in early and late ends may be
used as a defensive technique to prevent high scoring ends. The association
with relative team strength indicates that stronger teams more consistently
translate open-sheet conditions into multi-point scores, even after adjusting
for shot-level variables.

Overall, these findings suggest that the main driver of Power Play conversion
is execution quality across the first three hammer shots, while call
differences appear most relevant early and mid-end when teams are establishing
or reshaping scoring structure. Future work that incorporates spatial measures
of board state and models the full end score could translate these associations
into more explicit situation-specific tactical guidelines.

\bibliographystyle{apalike} \clearpage
\bibliography{refs.bib}

\end{document}